\documentclass[10pt,a4paper,landscape]{article}

% Pakete für kompaktes Layout und deutsche Umlaute
\usepackage[utf8]{inputenc}
\usepackage[ngerman]{babel}
\usepackage[margin=1cm]{geometry}
\usepackage{multicol}
\usepackage{amsmath}
\usepackage{amsfonts}
\usepackage{amssymb}
\usepackage{array}
\usepackage{siunitx}
\usepackage{makecell}
\usepackage{graphicx}

% Kompakte Darstellung
\setlength{\parindent}{0pt}
\setlength{\parskip}{0.5ex}
\setlength{\columnsep}{1cm}

% Überschriften anpassen
\usepackage{titlesec}
\titleformat{\section}{\normalfont\Large\bfseries}{\thesection}{1em}{}
\titlespacing*{\section}{0pt}{1ex}{0.5ex}
\titleformat{\subsection}{\normalfont\large\bfseries}{\thesubsection}{1em}{}
\titlespacing*{\subsection}{0pt}{0.5ex}{0.25ex}

% Kopf- und Fußzeile entfernen
\pagestyle{empty}

% Redefine section commands to use less space
\makeatletter
\renewcommand{\section}{\@startsection{section}{1}{0mm}%
                                {-1ex plus -.5ex minus -.2ex}%
                                {0.5ex plus .2ex}%x
                                {\normalfont\large\bfseries}}
\renewcommand{\subsection}{\@startsection{subsection}{2}{0mm}%
                                {-1explus -.5ex minus -.2ex}%
                                {0.5ex plus .2ex}%
                                {\normalfont\normalsize\bfseries}}

\renewcommand{\subsubsection}{\@startsection{subsubsection}{3}{0mm}%
                                {-1ex plus -.5ex minus -.2ex}%
                                {0.5ex plus .2ex}%
                                {\normalfont\small\bfseries}}
\makeatother

\setlength{\parindent}{0pt}
\setlength{\parskip}{0pt plus 0.5ex}

\begin{document}

\begin{center}
  \Large\textbf{Grundlagen der Elektro- und Digitaltechnik}
\end{center}

\begin{multicols*}{3}
  \setlength{\premulticols}{1pt}
  \setlength{\postmulticols}{1pt}
  \setlength{\multicolsep}{1pt}
  \setlength{\columnsep}{2pt}

  \section*{Physikalische Grundlagen}
  \subsection*{Wichtige Größen}
  \begin{itemize}
      \item Spannung (U) [Volt, V] = $\frac{J}{C}$
      \item Strom (I) [Ampere, A] = $\frac{C}{s}$
      \item Widerstand (R) [Ohm, $\Omega$] = $\frac{V}{A}$
      \item Leistung (P) [Watt, W] = $V \cdot A$
      \item Kapazität (C) [Farad, F] = $\frac{A \cdot s}{V}$
      \item Induktivität (L) [Henry, H] = $\frac{V \cdot s}{A}$
      \item Ladung (q) [Coulomb, C] = $A \cdot s$
      \item Kraft (F) [Newton, N] = $kg \cdot \frac{m}{s^2}$
      \item Energie = Arbeit (E = W) [Joule, J] = $N \cdot m$
      \item Frequenz (f) [Hertz, Hz] = $\frac{1}{s}$
      \item Magnetische Flussdichte (B) [Tesla, T] = $\frac{N}{A \cdot m}$
  \end{itemize}
  \subsection*{Wichtige Konstanten}
  {\small
    \begin{itemize}
        \item Erdbeschleunigung (g) = \SI{9.81}{\frac{m}{s^2}}
        \item Lichtgeschwindigkeit (c) = \SI{3e8}{\frac{m}{s}}
        \item Elektrische Feldkonstante ($\varepsilon_0$) = \SI{8.854e-12}{\frac{C^2}{N m^2}}
        \item Magnetische Feldkonstante ($\mu_0$) = \SI{1.26e-6}{\frac{Tm}{A}}
        \item Gravitationskonstante ($\gamma$) = \SI{6.67430e-11}{\frac{m^3}{kg s^2}}
        \item Elementarladung (e) = \SI{1.602176634e-19}{C}
    \end{itemize}
  }
  \subsection*{Beschleunigung}
  
  \begin{center}
    \begin{tabular}{|l|c|c|c|c|}
    \hline
     & \textbf{t} & \textbf{s} & \textbf{v} & \textbf{a} \\
    \hline
    \textbf{t} & - & $s = \frac{v^2}{2a}$ & $v = \sqrt{2as}$ & $a = \frac{v^2}{2s}$ \\
    \hline
    \textbf{s} & $t = \frac{v}{a}$ & - & $v = at$ & $a = \frac{v}{t}$ \\
    \hline
    \textbf{v} & $t = \sqrt{\frac{2s}{a}}$ & $s = \frac{at^2}{2}$ & - & $a = \frac{2s}{t^2}$ \\
    \hline
    \textbf{a} & $t = \frac{2s}{v}$ & $s = \frac{vt}{2}$ & $v = \frac{2s}{t}$ & - \\
    \hline
    \end{tabular}
  \end{center}

  \subsection*{Wurf mit Vektoren}
  \begin{itemize}
    \item $\vec{v}(t) = \begin{pmatrix} v_x \\ v_y \\ v_z - g \cdot t \end{pmatrix}$
    \item $\vec{s}(t) = \begin{pmatrix} s_x + v_x \cdot t \\ s_y + v_y \cdot t \\ s_z + v_z \cdot t - \frac{1}{2} g \cdot t^2 \end{pmatrix}$
  \end{itemize}

  \columnbreak

  \subsection*{Kräfte}
  \subsubsection*{Gewichtkraft}
  \begin{center}
    \begin{tabular}{|c|c|}
      \hline
      \textbf{Vektor} & \textbf{Zahlwert} \\
      \hline
      $\vec{F_G} = \begin{pmatrix} 0 \\ 0 \\ -m \cdot g \end{pmatrix}$ & 
      $F_G = m \cdot g$ \\
      \hline
    \end{tabular} \\[1em]
    m: Masse [kg], h: Höhe [m]
  \end{center}

  \subsubsection*{Federkraft}
  \begin{center}
    \begin{tabular}{|c|c|}
      \hline
      \textbf{Vektor} & \textbf{Zahlwert} \\
      \hline
      $\vec{F_s} = -k \cdot (|\vec{x}| - L) \cdot \frac{\vec{x}}{|\vec{x}|}$ & 
      $F_s = -k \cdot (x - L)$ \\
      \hline
    \end{tabular} \\[1em]
    $k$: Federkonstante \(\left[\frac{N}{m}\right]\), L: Ruhelänge [m], \\ x: Auslenkung [m]
  \end{center}

  \subsubsection*{Gravitationskraft zwischen Massen}
  \begin{center}
    \begin{tabular}{|c|c|}
      \hline
      \textbf{Vektor} & \textbf{Zahlwert} \\
      \hline
      $\vec{F_{12}} = -\gamma \frac{m_1 m_2}{|\vec{r}_1 - \vec{r}_2|^2} \frac{\vec{r}_1 - \vec{r}_2}{|\vec{r}_1 - \vec{r}_2|}$ & 
      $F_{12} = - \gamma \frac{m_1 m_2}{r^2}$ \\
      \hline
    \end{tabular} \\[1em]
    $m_1, m_2$: Massen [kg], \\
    $r_1, r_2$: Orte der Massen [m], r: Abstand der Massen [m] \\
    $\vec{n}_{12} = \frac{\vec{r}_1 - \vec{r}_2}{|\vec{r}_1 - \vec{r}_2|}$: Einheitsvektor von Masse 2 zu Masse 1
  \end{center}
    
  \subsubsection*{Kraft zwischen Ladungen (Coulomb-Kraft)}
  \begin{center}
    \begin{tabular}{|c|c|}
      \hline
      \textbf{Vektor} & \textbf{Zahlwert} \\
      \hline
      $\vec{F_{12}} = \frac{1}{4 \pi \varepsilon_0} \frac{q_1 q_2}{|\vec{r}_1 - \vec{r}_2|^2} \frac{\vec{r}_1 - \vec{r}_2}{|\vec{r}_1 - \vec{r}_2|}$ & 
      $F_{12} = \frac{1}{4 \pi \varepsilon_0} \frac{q_1 q_2}{r^2}$ \\
      \hline
    \end{tabular} \\[1em]
    $q_1, q_2$: Ladungen [C], \\
    $r_1, r_2$: Orte der Ladungen [m] \\
  \end{center}

  \subsection*{Energie}
  \begin{itemize}
    \item Potentielle Energie: $E_{pot} = m \cdot g \cdot h$
    \item Kinetische Energie: $E_{kin} = \frac{1}{2} m \cdot v^2$
    \item Federenergie: $E_s = \frac{1}{2} k (x - L)^2$
    \item Potentielle Energie einer Ladung bei einer Spannung: $E_{pot.el} = Uq$
    \item Potential: $\varphi = \frac{\text{Potentielle Energie der Menge X}}{\text{Menge X}}$
  \end{itemize}
  
  \columnbreak

  \section*{Veränderungsraten}
  \begin{itemize}
    \item $v(t) = \frac{ds(t)}{dt} = \int a(t) dt$
    \item $a(t) = \frac{dv(t)}{dt} = \frac{d^2 s(t)}{dt^2}$
    \item $s(t) = \int v(t) dt$
    \item $I(t) = \frac{dq(t)}{dt}$
  \end{itemize}

  \section*{Elektrische Grundlagen}

  \subsection*{Ohmsches Gesetz}
  \begin{itemize}
    \item $U = U_R = R \cdot I$
    \item $R = \rho \cdot \frac{l}{A}$ 
      \begin{itemize}
        \item $\rho$: spezifischer Widerstand \([\Omega m]\)
        \item $l$: Länge des Leiters [m]
        \item $A$: Querschnittsfläche des Leiters \([m^2]\)
      \end{itemize}
    \item $P = U \cdot I = R \cdot I^2 = \frac{U^2}{R}$
  \end{itemize}

  \subsection*{Knotenregel (1. Kirchhoffsche Regel)}
  Ein Knoten ist ein Verbindungspunkt mehrerer Leiter.
  \begin{equation*}
    \sum I_{ein} = \sum I_{aus}
  \end{equation*}
  
  \subsection*{Maschenregel (2. Kirchhoffsche Regel)}
  Eine Masche ist ein geschlossener Weg in einem Netzwerk.
  \begin{equation*}
    \sum U_{Quelle} = \sum U_{Verbraucher}
  \end{equation*}
  
  \subsection*{Reihenschaltung}
  \begin{itemize}
    \item $R_{gesamt} = R_1 + R_2 + \ldots + R_n$
    \item $I_{gesamt} = I_1 = I_2 = \ldots = I_n$
    \item $U_{gesamt} = U_1 + U_2 + \ldots + U_n$
  \end{itemize}

  \subsection*{Parallelschaltung}
  \begin{itemize}
    \item $\frac{1}{R_{gesamt}} = \frac{1}{R_1} + \frac{1}{R_2} + \ldots + \frac{1}{R_n}$
    \item $I_{gesamt} = I_1 + I_2 + \ldots + I_n$
    \item $U_{gesamt} = U_1 = U_2 = \ldots = U_n$
  \end{itemize}

  \subsection*{Lastwiederstand}
  \begin{equation*}
    P_{last} = U_0^2 \cdot \frac{R_{last}}{(R_i + R_{last})^2}
  \end{equation*}

  \columnbreak

  \subsection*{Kondensator}
  \begin{itemize}
    \item $C = \frac{q}{U}$ (Kapazität) [F]
    \item $E_{el} = \frac{1}{2} C U^2$
    \item $I(t) = \frac{U_0 - U_C(t)}{R} = \frac{U_0}{R} e^{-\frac{t}{\tau}}$
    \item $U_C(t) = U_0 \left(1 - e^{-\frac{t}{\tau}} \right)$
    \item $X_C = \frac{1}{\omega C}$ (Kapazitiver Blindwiderstand) [$\Omega$]
      \begin{itemize}
        \item Kann nicht direkt mit R addiert werden
      \end{itemize}
    \item $\tau = R \cdot C$ (Zeitkonstante) [s]
    \item $\omega = 2 \pi f$ (Kreisfrequenz) [Hz]
    \item $f_g = \frac{1}{2 \pi \tau}$ (Grenzfrequenz) [Hz]
  \end{itemize}

  \subsection*{Induktivität}
  \begin{itemize}
    \item $U_L(t) = L \frac{dI(t)}{dt} = U_0 e^{-\frac{t}{\tau}}$
    \item $E_{mag} = \frac{1}{2} L I^2$
    \item $I(t) = I_0 \left(1 - e^{-\frac{t}{\tau}} \right)$
    \item $X_L = \omega L$ (Induktiver Blindwiderstand) [$\Omega$]
      \begin{itemize}
        \item Kann nicht direkt mit R addiert werden
      \end{itemize}
    \item $\tau = \frac{L}{R}$ (Zeitkonstante) [s]
    \item $\omega = 2 \pi f$ (Kreisfrequenz) [Hz]
    \item $f_g = \frac{1}{2 \pi \tau}$ (Grenzfrequenz) [Hz]
  \end{itemize}

  \subsection*{Schwingkreis}
  \begin{itemize}
    \item $f = \frac{1}{2 \pi} \sqrt{\frac{1}{L C} - \frac{R^2}{4 L^2}}$ (Frequenz) [Hz]
    \item $T = \frac{1}{f}$ (Periodendauer) [s]
    \item $\tau = \frac{2L}{R}$ (Abklingzeit / Zeitkonstante) [s]
    \item Wenn $R^2 < \frac{4L}{C}$: Unterkritisch
    \item Wenn $R^2 = \frac{4L}{C}$: Kritisch
      \begin{itemize}
        \item Schnellstes Abklingen ohne Überschwingen
      \end{itemize}
    \item Wenn $R^2 > \frac{4L}{C}$: Überkritisch
      \begin{itemize}
        \item Exponentielles Abklingen ohne Schwingung
      \end{itemize}
  \end{itemize}
  
  \columnbreak

  \section*{Signale}
  \subsection*{Grundlegende Begriffe}
  \begin{itemize}
    \item Amplitude (A): Maximaler Ausschlag eines Signals
    \item Periodendauer (T): Zeit für eine vollständige Schwingung [s]
    \item Frequenz (f): Anzahl der Schwingungen pro Sekunde [Hz]
      \begin{equation*}
        f_n = n \cdot \frac{1}{T}
      \end{equation*}
    \item Kreisfrequenz ($\omega$): Winkelgeschwindigkeit [rad/s]
      \begin{equation*}
        \omega_n = 2 \pi f_n
      \end{equation*}
    \item Phasenverschiebung ($\varphi$): Zeitliche Verschiebung eines Signals [rad]
      \begin{equation*}
        \varphi = \omega \cdot t_{Verschiebung}
      \end{equation*}
    \item Serieschaltung: Signale werden addiert
    \item Parallelschaltung: Signale werden gemittelt
  \end{itemize}

  \subsection*{Wichtige Signale}
  \begin{itemize}
    \item Sinus: $y(t) = A \cdot \sin(\omega t + \varphi)$
    \item Rechteck: $y(t) = \begin{cases} A & 0 \leq t < \frac{T}{2} \\ -A & \frac{T}{2} \leq t < T \end{cases}$
    \item Dreieck: $y(t) = \begin{cases} \frac{4A}{T} t - A & 0 \leq t < \frac{T}{2} \\ -\frac{4A}{T} t + 3A & \frac{T}{2} \leq t < T \end{cases}$
    \item Sägezahn: $y(t) = \frac{2A}{T} t - A$
  \end{itemize}

  \subsection*{Fourierzerlegung}
  \begin{itemize}
    \item Fourierreihe: 
      \begin{equation*}
        g(t) = \frac{a_0}{2} + \sum_{n=1}^{\infty} \left( a_n \cos(\omega_n t) + b_n \sin(\omega_n t) \right)
      \end{equation*}
    \item Fourierkoeffizienten:
      \begin{equation*}
        a_n = \frac{2}{T} \int_{-\frac{T}{2}}^{\frac{T}{2}} g(t) \cos(\omega_n t) dt
      \end{equation*}
      \begin{equation*}
        b_n = \frac{2}{T} \int_{-\frac{T}{2}}^{\frac{T}{2}} g(t) \sin(\omega_n t) dt
      \end{equation*}
  \end{itemize}

  \columnbreak

  \subsection*{Amplituden- und Phasendarstellung}
  \begin{itemize}
    \item Amplituden- und Phasendarstellung:
      \begin{equation*}
        g(t) = A_n \cdot \sin(\omega_n t + \varphi_n)
      \end{equation*}
    \item Zusammenhang der Darstellungen:
      \begin{equation*}
        A_n = \sqrt{a_n^2 + b_n^2}
      \end{equation*}
      \begin{equation*}
        \varphi_n = \arctan\left(\frac{a_n}{b_n}\right) = \arcsin\left(\frac{b_n}{A_n}\right) = \arccos\left(\frac{a_n}{A_n}\right)
      \end{equation*}
  \end{itemize}

  \subsection*{Inverse Fouriertransform}
  \begin{equation*}
    g(t) = \frac{1}{2\pi} \int_{-\infty}^{\infty} G(\omega) e^{j \omega t} d\omega
  \end{equation*}

  \columnbreak

  \section*{Digitaltechnik}
  \subsection*{Halbleiter}
  \begin{itemize}
    \item n-Halbleiter: Überschuss an Elektronen (Donatoren)
    \item p-Halbleiter: Überschuss an Löchern (Akzeptoren)
    \item pn-Übergang: Verbindung von p- und n-Halbleiter
  \end{itemize}
  \subsubsection*{Dotierung / Doping}
  \begin{itemize}
    \item n-Dotierung: Zugabe von Elementen der 5. Hauptgruppe (z.B. Phosphor, Arsen)
    \item p-Dotierung: Zugabe von Elementen der 3. Hauptgruppe (z.B. Bor, Gallium)
  \end{itemize}

  \subsection*{Halbleiterbauelemente}
  \subsubsection*{Diode}
  \begin{itemize}
    \item Leitet Strom in Durchlassrichtung (Anode zu Kathode)
    \item Sperrt Strom in Sperrrichtung (Kathode zu Anode)
    \item Kennlinie: Exponentielles Verhalten
  \end{itemize}

  \subsubsection*{Transistor (MOSFET)}
  \begin{itemize}
    \item Steuerung des Stromflusses zwischen Drain und Source durch Spannung am Gate
    \item Leitet Strom, wenn Spannung am Gate über Schwellenspannung liegt
  \end{itemize}

  \subsubsection*{Logische Gatter}
  \includegraphics[width=\columnwidth]{images/components.png}

  \columnbreak

  \subsubsection*{RS-Flip-Flop}
  \includegraphics[width=0.7 \columnwidth]{images/rs-flipflop.png}
  \begin{itemize}
    \item Setzt Ausgang Q auf 1, wenn S=1 und R=0
    \item Setzt Ausgang Q auf 0, wenn S=0 und R=1
    \item Beibehaltung des Zustands, wenn S=0 und R=0
    \item Ungültiger Zustand, wenn S=1 und R=1
  \end{itemize}

  \subsubsection*{D-Flip-Flop}
  \begin{minipage}{0.35\linewidth}
    \includegraphics[width=\linewidth]{images/d-flipflop.png}
  \end{minipage} \hfill
  \begin{minipage}{0.6\linewidth}
    \begin{itemize}
      \item Überträgt Eingang D auf Ausgang Q bei Taktflanke
      \item Verhindert ungewollte Änderungen des Ausgangs
    \end{itemize}
  \end{minipage}

  \subsubsection*{JK-Flip-Flop}
  \begin{minipage}{0.35\linewidth}
    \includegraphics[width=\linewidth]{images/jk-flipflop.png}
    
  \end{minipage} \hfill
  \begin{minipage}{0.6\linewidth}
    \begin{itemize}
      \item Setzt Ausgang Q auf 1, wenn J=1 und K=0
      \item Setzt Ausgang Q auf 0, wenn J=0 und K=1
      \item Toggle-Zustand, wenn J=1 und K=1
      \item Beibehaltung des Zustands, wenn J=0 und K=0
    \end{itemize}
  \end{minipage}

  \subsection*{Parasiteneffekte}
  \begin{itemize}
    \item Kapazitive Kopplung: Unerwünschte elektrische Verbindung zwischen Leitungen
    \item Induktive Kopplung: Unerwünschte magnetische Verbindung zwischen Leitungen
    \item Übersprechen: Signalübertragung von einer Leitung auf eine andere
    \item Signalreflexion: Rückkehr eines Signals aufgrund von Impedanzänderungen
  \end{itemize}

  \columnbreak

  \subsection*{Sensoren}

  \columnbreak
  \section*{Elektrische und Magnetische Felder}
  \subsection*{Coulomb-Gesetz}
  Siehe Kraft zwischen Ladungen Seite 1.
  \begin{equation*}
    \vec{F}_{12} = \vec{E} \cdot q_p
  \end{equation*}
  \subsection*{Elektrisches Feld}
  \begin{itemize}
    \item Elektrische Feldstärke: 
      \begin{equation*}
        \vec{E} = \frac{\vec{F}}{q} = \frac{1}{4 \pi \varepsilon_0} \frac{q}{|\vec{r} - \vec{r}_q|^2} \frac{\vec{r} - \vec{r}_q}{|\vec{r} - \vec{r}_q|} \quad \left[\frac{N}{C}\right] = \left[\frac{V}{m}\right]
      \end{equation*}
    \item Elektrisches Potential:
      \begin{equation*}
        \varphi = \frac{E_{pot}}{q} \quad [V]
      \end{equation*}
    \item Spannung zwischen zwei Punkten:
      \begin{equation*}
        U_{AB} = \varphi_A - \varphi_B = \int_A^B \vec{E} \cdot d\vec{s}
      \end{equation*}
  \end{itemize}
  \subsection*{Magnetisches Feld}
  \begin{itemize}
    \item Magnetische Flussdichte:
      \begin{equation*}
        \vec{B} = \frac{\vec{F}}{q \cdot \vec{v}} \quad [T]
      \end{equation*}
    \item Lorentzkraft:
      \begin{equation*}
        \vec{F_L} = q \cdot \vec{v} \times \vec{B}
      \end{equation*}
    \item Wenn $\vec{v} \perp \vec{B}$ gilt:
      \begin{equation*}
        m = \frac{q \cdot |\vec{B}| \cdot r}{v} \quad \text{(Masse des Teilchens)} [kg]
      \end{equation*}
    \item Strom ist immer senkrecht zum Magnetfeld
  \end{itemize}

  \subsection*{Elektromagnetische Kraft und Feldenergie}
  \begin{itemize}
    \item Kraft auf eine Ladung im elektromagnetischen Feld:
      \begin{equation*}
      \vec{F}_{elmag} = q(\vec{E}(\vec{r}) + \vec{v} \times \vec{B}(\vec{r})) \quad [N]
      \end{equation*}
    \item Energiedichte des elektromagnetischen Feldes:
      \begin{equation*}
        w = \frac{\varepsilon_0}{2} |\vec{E}|^2 + \frac{\varepsilon_0 c^2}{2} |\vec{B}|^2 \quad \left[\frac{J}{m^3}\right]
      \end{equation*}
    \item Energie in einem Volumen V:
      \begin{equation*}
        \varepsilon = \int_V w \, dV \quad [J]
      \end{equation*}
  \end{itemize}

  \columnbreak

  \section*{Elektrodynamik}
  \subsection*{Der Fluss}
  \begin{itemize}
    \item Elektrischer Fluss:
      \begin{equation*}
        \Phi_E = \frac{q}{\varepsilon_0} \quad \left[\frac{N m^2}{C}\right]
      \end{equation*}
    \item Wenn $\vec{E}$ homogen und senkrecht zur Kugeöberfläche ist
      \begin{equation*}
        \Phi_E = 4 \pi r^2 \cdot |\vec{E}| = \frac{q}{\varepsilon_0}
      \end{equation*}
      \begin{equation*}
        |\vec{E}| = \frac{q}{4 \pi \varepsilon_0 r^2}
      \end{equation*}
      \begin{equation*}
        \vec{E} = \frac{q}{4 \pi \varepsilon_0 r^2} \cdot \frac{\vec{r}}{|\vec{r}|}
      \end{equation*}
  \end{itemize}

  \subsection*{Integration entlang von Linien}
  \begin{itemize}
    \item Arbeit eines Kraftfeldes entlang einer Linie:
      \begin{equation*}
        W = \int_A^B \vec{F}(\vec{r}) \cdot d\vec{s} \quad [J]
      \end{equation*}
    \item Spannung zwischen zwei Punkten:
      \begin{equation*}
        U_{AB} = \int_A^B \vec{E}(\vec{r}) \cdot d\vec{s} \quad [V]
      \end{equation*}
  \end{itemize}

  \subsection*{Magnetfeld eines stromdurchflossenen Leiters}
  \begin{itemize}
    \item Magnetfeld um einen gestreckten Leiter:
      \begin{equation*}
        |\vec{B}(\vec{r})| = \frac{\mu_0 I}{2 \pi r} \quad [T]
      \end{equation*}
    \item Magnetfeld von Spulen:
      \begin{equation*}
        |\vec{B}| = \mu_0 \mu_r \frac{N}{L} I \quad [T]
      \end{equation*}
      \begin{center}
        N: Windungszahl, L: Länge der Spule [m], \\
        $\mu_r$: relative Permeabilität des Spulenkerns
      \end{center}
  \end{itemize}

  \section*{Induktion}
  \subsection*{Induktionsgesetz}
  \begin{itemize}
    \item Induzierte Spannung in einer Leiterschleife:
      \begin{equation*}
        U_{ind} = -\frac{d\Phi}{dt} \quad [V]
      \end{equation*}
    \item Magnetischer Fluss im Dynamo:
      \begin{equation*}
        \Phi = A_1 \cdot E = A_2 \cdot E \cdot \cos(\theta)
      \end{equation*}
      \begin{center}
        $A_1$: Fläche der Spule [m$^2$], \\
        $A_2$: Fläche des Rotors [m$^2$], \\
        $\theta$: Winkel zwischen Magnetfeld und Flächennormalen $\theta = \omega t$
      \end{center}
      \item Induzierte Spannung im Dynamo:
      \begin{equation*}
        U_{ind} = A \cdot \omega \cdot \sin(\omega t) \quad [V]
      \end{equation*}
  \end{itemize}

  \subsection*{Effektivwert}
  \begin{itemize}
    \item Effektivwert einer Sinusspannung:
      \begin{equation*}
        U_{eff} = \frac{U_{\max}}{\sqrt{2}} \quad [V]
      \end{equation*}
  \end{itemize}

  \subsection*{Senden und Empfangen von Signalen}
  \begin{itemize}
    \item Antennenlänge für eine Frequenz:
      \begin{equation*}
        l_{Ant} = \frac{c}{f} \quad [m]
      \end{equation*}
    \item Empfangene Leistung:
      \begin{equation*}
        P_{empfangen} = P_{senden} \cdot \left( \frac{l_{Ant}}{4 \pi r} \right)^2 \quad [W]
      \end{equation*}
      \begin{center}
        r: Abstand zwischen Sender und Empfänger [m]
      \end{center}
  \end{itemize}

  \subsection*{Transformator}
  \begin{itemize}
    \item Spannungsverhältnis(N=Windungszahl):
      \begin{equation*}
        \frac{U_1}{U_2} = \frac{N_1}{N_2}
      \end{equation*}
    \item Stromverhältnis:
      \begin{equation*}
        \frac{I_1}{I_2} = \frac{N_2}{N_1}
      \end{equation*}
    \item Leistung:
      \begin{equation*}
        P_1 = P_2
      \end{equation*}
  \end{itemize}

  \pagebreak
\end{multicols*}
\end{document}
