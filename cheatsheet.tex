\documentclass[10pt,a4paper,landscape]{article}

% Pakete für kompaktes Layout und deutsche Umlaute
\usepackage[utf8]{inputenc}
\usepackage[ngerman]{babel}
\usepackage[margin=1cm]{geometry}
\usepackage{multicol}
\usepackage{amsmath}
\usepackage{amsfonts}
\usepackage{amssymb}
\usepackage{array}
\usepackage{booktabs}
\usepackage{siunitx}

% Kompakte Darstellung
\setlength{\parindent}{0pt}
\setlength{\parskip}{0.5ex}
\setlength{\columnsep}{1cm}

% Überschriften anpassen
\usepackage{titlesec}
\titleformat{\section}{\normalfont\Large\bfseries}{\thesection}{1em}{}
\titlespacing*{\section}{0pt}{1ex}{0.5ex}
\titleformat{\subsection}{\normalfont\large\bfseries}{\thesubsection}{1em}{}
\titlespacing*{\subsection}{0pt}{0.5ex}{0.25ex}

% Kopf- und Fußzeile entfernen
\pagestyle{empty}

\begin{document}

\begin{center}
    \Large\textbf{Physik Spickzettel - Beschleunigung}
\end{center}

\begin{multicols}{2}

\section{Grundlagen der Beschleunigung}

Die Beschleunigung ist die Änderung der Geschwindigkeit pro Zeiteinheit.

\subsection{Beschleunigungsformeln}

\begin{center}
\begin{tabular}{|p{1.5cm}|p{2.5cm}|p{2.5cm}|p{2.5cm}|p{2.5cm}|}
\hline
 & \textbf{t} & \textbf{s} & \textbf{v} & \textbf{a} \\
\hline
\textbf{t} & - & $s = \frac{v^2}{2a}$ & $v = \sqrt{2as}$ & $a = \frac{v^2}{2s}$ \\
\hline
\textbf{s} & $t = \frac{v}{a}$ & - & $v = at$ & $a = \frac{v}{t}$ \\
\hline
\textbf{v} & $t = \sqrt{\frac{2s}{a}}$ & $s = \frac{at^2}{2}$ & - & $a = \frac{2s}{t^2}$ \\
\hline
\textbf{a} & $t = \frac{2s}{v}$ & $s = \frac{vt}{2}$ & $v = \frac{2s}{t}$ & - \\
\hline
\end{tabular}
\end{center}

\subsection{Kraft}
Die Kraft ist das Produkt aus Masse und Beschleunigung:
\begin{equation}
F = m \cdot a
\end{equation}

\subsection{Spezielle Beschleunigungen}

\textbf{Kreisbewegung:}
\begin{align}
a_z &= \frac{v^2}{r} = \omega^2 r \\
a_t &= \frac{dv}{dt} = r \alpha
\end{align}

\textbf{Harmonische Schwingung:}
\begin{align}
a &= -\omega^2 x \\
a_{max} &= \omega^2 A
\end{align}

\section{Energie}
\begin{align}
E_{kin} &= \frac{1}{2}mv^2 \\
E_{pot} &= mgh \\
E_{el} &= \frac{1}{2}kx^2 \\
E_{Feder} &= \frac{1}{2}kx^2 \\
E_{reibung} &= F_{reib} \cdot s \\
\end{align}



\section{Wichtige Konstanten}

\begin{itemize}
    \item Erdbeschleunigung: $g = 9{,}81 \, \frac{m}{s^2}$
    \item Fallbeschleunigung Mond: $g_{Mond} = 1{,}62 \, \frac{m}{s^2}$
\end{itemize}

\section{Einheiten und Umrechnungen}

\begin{itemize}
    \item $1 \, \frac{m}{s^2} = 100 \, \frac{cm}{s^2}$
    \item $1 \, g = 9{,}81 \, \frac{m}{s^2}$ (Erdbeschleunigung als Einheit)
\end{itemize}

\columnbreak

\section{Kinematische Gleichungen}

Für konstante Beschleunigung gelten die folgenden Zusammenhänge:

\begin{align}
v(t) &= v_0 + at \\
s(t) &= s_0 + v_0 t + \frac{1}{2}at^2 \\
v^2 &= v_0^2 + 2a(s - s_0) \\
\bar{v} &= \frac{s - s_0}{t} = \frac{v_0 + v}{2}
\end{align}

\section{Freier Fall}

Beim freien Fall wirkt nur die Gravitationskraft:

\begin{align}
v(t) &= gt \\
h(t) &= \frac{1}{2}gt^2 \\
v^2 &= 2gh \\
t_{Fall} &= \sqrt{\frac{2h}{g}}
\end{align}

\section{Schiefer Wurf}

\textbf{Horizontal:} $a_x = 0$
\begin{align}
v_x &= v_0 \cos(\alpha) \\
x(t) &= v_0 \cos(\alpha) \cdot t
\end{align}

\textbf{Vertikal:} $a_y = -g$
\begin{align}
v_y(t) &= v_0 \sin(\alpha) - gt \\
y(t) &= v_0 \sin(\alpha) \cdot t - \frac{1}{2}gt^2
\end{align}

\textbf{Wurfweite:} $R = \frac{v_0^2 \sin(2\alpha)}{g}$

\textbf{Maximale Höhe:} $h_{max} = \frac{v_0^2 \sin^2(\alpha)}{2g}$

\end{multicols}

\end{document}