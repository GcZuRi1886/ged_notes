\documentclass[10pt,a4paper,landscape]{article}

% Pakete für kompaktes Layout und deutsche Umlaute
\usepackage[utf8]{inputenc}
\usepackage[ngerman]{babel}
\usepackage[margin=1cm]{geometry}
\usepackage{multicol}
\usepackage{amsmath}
\usepackage{amsfonts}
\usepackage{amssymb}
\usepackage{array}
\usepackage{siunitx}
\usepackage{makecell}

% Kompakte Darstellung
\setlength{\parindent}{0pt}
\setlength{\parskip}{0.5ex}
\setlength{\columnsep}{1cm}

% Überschriften anpassen
\usepackage{titlesec}
\titleformat{\section}{\normalfont\Large\bfseries}{\thesection}{1em}{}
\titlespacing*{\section}{0pt}{1ex}{0.5ex}
\titleformat{\subsection}{\normalfont\large\bfseries}{\thesubsection}{1em}{}
\titlespacing*{\subsection}{0pt}{0.5ex}{0.25ex}

% Kopf- und Fußzeile entfernen
\pagestyle{empty}

% Redefine section commands to use less space
\makeatletter
\renewcommand{\section}{\@startsection{section}{1}{0mm}%
                                {-1ex plus -.5ex minus -.2ex}%
                                {0.5ex plus .2ex}%x
                                {\normalfont\large\bfseries}}
\renewcommand{\subsection}{\@startsection{subsection}{2}{0mm}%
                                {-1explus -.5ex minus -.2ex}%
                                {0.5ex plus .2ex}%
                                {\normalfont\normalsize\bfseries}}

\renewcommand{\subsubsection}{\@startsection{subsubsection}{3}{0mm}%
                                {-1ex plus -.5ex minus -.2ex}%
                                {0.5ex plus .2ex}%
                                {\normalfont\small\bfseries}}
\makeatother

\setlength{\parindent}{0pt}
\setlength{\parskip}{0pt plus 0.5ex}

\begin{document}

\begin{center}
  \Large\textbf{Grundlagen der Elektro- und Digitaltechnik}
\end{center}

\begin{multicols*}{3}
  \setlength{\premulticols}{1pt}
  \setlength{\postmulticols}{1pt}
  \setlength{\multicolsep}{1pt}
  \setlength{\columnsep}{2pt}

  \section*{Physikalische Grundlagen}
  \subsection*{Wichtige Größen}
  \begin{itemize}
      \item Spannung (U) [Volt, V] = $\frac{J}{C}$
      \item Strom (I) [Ampere, A] = $\frac{C}{s}$
      \item Widerstand (R) [Ohm, $\Omega$] = $\frac{V}{A}$
      \item Leistung (P) [Watt, W] = $V \cdot A$
      \item Kapazität (C) [Farad, F] = $\frac{A \cdot s}{V}$
      \item Induktivität (L) [Henry, H] = $\frac{V \cdot s}{A}$
      \item Ladung (q) [Coulomb, C] = $A \cdot s$
      \item Kraft (F) [Newton, N] = $kg \cdot \frac{m}{s^2}$
      \item Energie = Arbeit (E = W) [Joule, J] = $N \cdot m$
      \item Frequenz (f) [Hertz, Hz] = $\frac{1}{s}$
      \item Erdbeschleunigung (g) = \SI{9.81}{\frac{m}{s^2}}
      \item Lichtgeschwindigkeit (c) = \SI{3e8}{\frac{m}{s}}
      \item Elektrische Feldkonstante ($\varepsilon_0$) = \SI{8.854e-12}{\frac{C^2}{N m^2}}
      \item Gravitationskonstante ($\gamma$) = \SI{6.67430e-11}{\frac{m^3}{kg s^2}}
  \end{itemize}

  \subsection*{Beschleunigung}
  
  \begin{center}
    \begin{tabular}{|l|c|c|c|c|}
    \hline
     & \textbf{t} & \textbf{s} & \textbf{v} & \textbf{a} \\
    \hline
    \textbf{t} & - & $s = \frac{v^2}{2a}$ & $v = \sqrt{2as}$ & $a = \frac{v^2}{2s}$ \\
    \hline
    \textbf{s} & $t = \frac{v}{a}$ & - & $v = at$ & $a = \frac{v}{t}$ \\
    \hline
    \textbf{v} & $t = \sqrt{\frac{2s}{a}}$ & $s = \frac{at^2}{2}$ & - & $a = \frac{2s}{t^2}$ \\
    \hline
    \textbf{a} & $t = \frac{2s}{v}$ & $s = \frac{vt}{2}$ & $v = \frac{2s}{t}$ & - \\
    \hline
    \end{tabular}
  \end{center}

  \subsection*{Wurf mit Vektoren}
  \begin{itemize}
    \item $\vec{v}(t) = \begin{pmatrix} v_x \\ v_y \\ v_z - g \cdot t \end{pmatrix}$
    \item $\vec{s}(t) = \begin{pmatrix} s_x + v_x \cdot t \\ s_y + v_y \cdot t \\ s_z + v_z \cdot t - \frac{1}{2} g \cdot t^2 \end{pmatrix}$
  \end{itemize}

  \columnbreak

  \subsection*{Kräfte}
  \subsubsection*{Gewichtkraft}
  \begin{center}
    \begin{tabular}{|c|c|}
      \hline
      \textbf{Vektor} & \textbf{Zahlwert} \\
      \hline
      $\vec{F_G} = \begin{pmatrix} 0 \\ 0 \\ -m \cdot g \end{pmatrix}$ & 
      $F_G = m \cdot g$ \\
      \hline
    \end{tabular} \\[1em]
    m: Masse [kg], h: Höhe [m]
  \end{center}

  \subsubsection*{Federkraft}
  \begin{center}
    \begin{tabular}{|c|c|}
      \hline
      \textbf{Vektor} & \textbf{Zahlwert} \\
      \hline
      $\vec{F_s} = -k \cdot (|\vec{x}| - L) \cdot \frac{\vec{x}}{|\vec{x}|}$ & 
      $F_s = -k \cdot (x - L)$ \\
      \hline
    \end{tabular} \\[1em]
    $k$: Federkonstante \(\left[\frac{N}{m}\right]\), L: Ruhelänge [m], \\ x: Auslenkung [m]
  \end{center}

  \subsubsection*{Gravitationskraft zwischen Massen}
  \begin{center}
    \begin{tabular}{|c|c|}
      \hline
      \textbf{Vektor} & \textbf{Zahlwert} \\
      \hline
      $\vec{F_{12}} = -\gamma \frac{m_1 m_2}{|\vec{r}_1 - \vec{r}_2|^2} \frac{\vec{r}_1 - \vec{r}_2}{|\vec{r}_1 - \vec{r}_2|}$ & 
      $F_{12} = - \gamma \frac{m_1 m_2}{r^2}$ \\
      \hline
    \end{tabular} \\[1em]
    $m_1, m_2$: Massen [kg], \\
    $r_1, r_2$: Orte der Massen [m], r: Abstand der Massen [m] \\
    $\vec{n}_{12} = \frac{\vec{r}_1 - \vec{r}_2}{|\vec{r}_1 - \vec{r}_2|}$: Einheitsvektor von Masse 2 zu Masse 1
  \end{center}
    
  \subsubsection*{Kraft zwischen Ladungen (Coulomb-Kraft)}
  \begin{center}
    \begin{tabular}{|c|c|}
      \hline
      \textbf{Vektor} & \textbf{Zahlwert} \\
      \hline
      $\vec{F_{12}} = \frac{1}{4 \pi \varepsilon_0} \frac{q_1 q_2}{|\vec{r}_1 - \vec{r}_2|^2} \frac{\vec{r}_1 - \vec{r}_2}{|\vec{r}_1 - \vec{r}_2|}$ & 
      $F_{12} = \frac{1}{4 \pi \varepsilon_0} \frac{q_1 q_2}{r^2}$ \\
      \hline
    \end{tabular} \\[1em]
    $q_1, q_2$: Ladungen [C], \\
    $r_1, r_2$: Orte der Ladungen [m] \\
    $\vec{n}_{12} = \frac{\vec{r}_1 - \vec{r}_2}{|\vec{r}_1 - \vec{r}_2|}$: Einheitsvec von Ladung 2 zu Ladung 1
  \end{center}

  \subsection*{Energie}
  \begin{itemize}
    \item Potentielle Energie: $E_{pot} = m \cdot g \cdot h$
    \item Kinetische Energie: $E_{kin} = \frac{1}{2} m \cdot v^2$
    \item Federenergie: $E_s = \frac{1}{2} k (x - L)^2$
    \item Potentielle Energie einer Ladung bei einer Spannung: $E_{pot.el} = Uq$
    \item Potential: $\varphi = \frac{\text{Potentielle Energie der Menge X}}{\text{Menge X}}$
  \end{itemize}
  
  \columnbreak

  \section*{Veränderungsraten}
  \begin{itemize}
    \item $v(t) = \frac{ds(t)}{dt}$
    \item $a(t) = \frac{dv(t)}{dt} = \frac{d^2 s(t)}{dt^2}$
    \item $s(t) = \int v(t) dt$
    \item $v(t) = \int a(t) dt$
    \item $I(t) = \frac{dq(t)}{dt}$
  \end{itemize}

  \section*{Elektrische Grundlagen}

  \subsection*{Ohmsches Gesetz}
  \begin{itemize}
    \item $U = U_R = R \cdot I$
    \item $R = \rho \cdot \frac{l}{A}$ 
      \begin{itemize}
        \item $\rho$: spezifischer Widerstand \([\Omega m]\)
        \item $l$: Länge des Leiters [m]
        \item $A$: Querschnittsfläche des Leiters \([m^2]\)
      \end{itemize}
    \item $P = U \cdot I = R \cdot I^2 = \frac{U^2}{R}$
  \end{itemize}

  \subsection*{Knotenregel (1. Kirchhoffsche Regel)}
  Ein Knoten ist ein Verbindungspunkt mehrerer Leiter.
  \begin{equation*}
    \sum I_{ein} = \sum I_{aus}
  \end{equation*}
  
  \subsection*{Maschenregel (2. Kirchhoffsche Regel)}
  Eine Masche ist ein geschlossener Weg in einem Netzwerk.
  \begin{equation*}
    \sum U_{Quelle} = \sum U_{Verbraucher}
  \end{equation*}
  
  \subsection*{Reihenschaltung}
  \begin{itemize}
    \item $R_{gesamt} = R_1 + R_2 + \ldots + R_n$
    \item $I_{gesamt} = I_1 = I_2 = \ldots = I_n$
    \item $U_{gesamt} = U_1 + U_2 + \ldots + U_n$
  \end{itemize}

  \subsection*{Parallelschaltung}
  \begin{itemize}
    \item $\frac{1}{R_{gesamt}} = \frac{1}{R_1} + \frac{1}{R_2} + \ldots + \frac{1}{R_n}$
    \item $I_{gesamt} = I_1 + I_2 + \ldots + I_n$
    \item $U_{gesamt} = U_1 = U_2 = \ldots = U_n$
  \end{itemize}

  \subsection*{Lastwiederstand}
  \begin{equation*}
    P_{last} = U_0^2 \cdot \frac{R_{last}}{(R_i + R_{last})^2}
  \end{equation*}
\end{multicols*}

\end{document}
